\documentclass[runningheads]{llncs}

\mode<presentation>
{
  \usetheme{Darmstadt}
  \useoutertheme{infolines}
}

\usepackage{alltt}

\usepackage{etex}         % to avoid compilation erros of xypic
\usepackage[curve]{xypic}
%\xyoption{pdf}
%\usepackage[pdf]{xy}


\usepackage[utf8]{inputenc}
\usepackage{proof}
\inferLineSkip=4pt  % increase spacing between lines; default is 2pt

\usepackage{amsmath}
\usepackage{amssymb}
\usepackage{fontawesome}
\usepackage{times}
\usepackage[T1]{fontenc}
\usepackage{listings}
\lstset{language=haskell,basicstyle=\ttfamily}

\usepackage{multicol}
\usepackage{mathpartir}
\usepackage{mathtools}
\usepackage{stmaryrd}
\usepackage{soul}
\usepackage{tikzsymbols}

% Delete this, if you do not want the table of contents to pop up at
% the beginning of each subsection:
\AtBeginSection[]
{
  \begin{frame}<beamer>
    \frametitle{Plan}
    \tableofcontents[sectionstyle=show/shaded,subsectionstyle=hide]
  \end{frame}
}

\AtBeginSubsection[]
{
  \begin{frame}<beamer>
    \frametitle{Plan}
    \tableofcontents[sectionstyle=show/hide,subsectionstyle=show/shaded/hide]
%    \tableofcontents[subsectionstyle=show/shaded/hide]
  \end{frame}
}

\setbeamertemplate{footline}
{%
%\begin{beamercolorbox}{section in head/foot}
%\vskip2pt\insertnavigation{\paperwidth}\vskip2pt
%\end{beamercolorbox}%
\insertpagenumber
\insertshorttitle[width={5cm},center]
\insertshortinstitute[width={3cm},center]
\insertshortdate[width={3cm},center]
}


% If you wish to uncover everything in a step-wise fashion, uncomment
% the following command: 

%\beamerdefaultoverlayspecification{<+->}

\usepackage{tikz}
\usetikzlibrary{trees}
\usetikzlibrary{arrows}
\usetikzlibrary{decorations.pathmorphing}
\usetikzlibrary{shapes.multipart}
\usetikzlibrary{shapes.geometric}
\usetikzlibrary{calc}
\usetikzlibrary{positioning} 
\usetikzlibrary{fit}
\usetikzlibrary{backgrounds}
\usetikzlibrary{automata}


%%% Local Variables: 
%%% mode: latex
%%% TeX-master: "main"
%%% End: 

%======================================================================
%%%%%%%%% DECLS / DEFNS %%%%%%%%%%%%%%%%%

% redefine existing green color
\newcommand{\brightgreen}[1]{{\color{green}#1}}
\definecolor{darkgreen}{rgb}{0,0.7,0}
\newcommand{\green}[1]{{\color{darkgreen}#1}}

\newcommand{\red}[1]{{\color{red}#1}}
\newcommand{\orange}[1]{{\color{orange}#1}}
\newcommand{\blue}[1]{{\color{blue}#1}}
\definecolor{grey}{rgb}{0.5,0.5,0.5}
\newcommand{\grey}[1]{{\color{grey}#1}}
\newcommand{\black}[1]{{\color{black}#1}}

% Logique
\newcommand{\IMPL}[0]{\longrightarrow}
\newcommand{\IMPLL}[0]{\Longrightarrow} % another implication, to make
                                % a difference with reduction relations
\newcommand{\AND}[0]{\wedge}
\newcommand{\OR}[0]{\vee}
\newcommand{\NOT}[0]{\neg}
\newcommand{\FALSE}[0]{\perp}
\newcommand{\TRUE}[0]{\top}
\newcommand{\IFF}[0]{\leftrightarrow}
\newcommand{\BIGAND}[1]{\bigwedge_{#1}}
\newcommand{\BIGOR}[1]{\bigvee_{#1}}
\newcommand{\BIGANDC}[2]{\bigwedge_{#1|#2}} % bigand with constraint
\newcommand{\BIGORC}[2]{\bigvee_{#1|#2}}    % bigor with constraint
\newcommand{\ufb}[0]{\mbox{$\stackrel{?}{=}$}} % unifiable

% semantique
\newcommand{\semtrans}[3]{\mbox{$\langle #1, #2 \rangle \rightarrow #3$}}
\newcommand{\statesel}[2]{\mbox{$#1 . #2$}}
\newcommand{\stateupd}[3]{\mbox{$#1 . #2 \leftarrow #3$}}


% Macros for mathematical notation

\def\N{{\mathbf N}}                      % ensemble des entiers naturels
\def\Z{{\mathbf Z}}                      % ensemble des entiers relatifs
\newcommand{\zmod}[1]{\mbox{$ \Z/{#1} \Z$}}  % anneau Z mod m


% WP-calcul
\newcommand{\wpform}[1]{\mbox{\{\( #1 \)\} }}
\newcommand{\subst}[3]{ #1 [ #2 \mbox{$\leftarrow\ $} #3]}
\newcommand{\pfp}[2]{\mbox{\emph{pfp}(\texttt{#1}, \mbox{$#2$})}}
\newcommand{\evb}[1]{\emph{évaluable}($#1$)}

% typage
\newcommand{\lange}[0]{{\cal L}_e}
\newcommand{\langf}[0]{{\cal L}_f}
\newcommand{\langs}[0]{{\cal L}_s}

% B method
\newcommand{\bsubst}[3]{ [ #2 := #3] #1 }

% reduction and evaluation
\newcommand{\tableaurule}[1]{{\xhookrightarrow[]{#1}}}

% special references
\newcommand{\secref}[1]{\S~\nameref{#1}, p.~\ref{#1}}
\newcommand{\transpref}[1]{transparent, p.~\ref{#1}}


% Other

\newcommand{\smalltalcq}[0]{{\small small}-t{$\cal ALCQ$}}
\newcommand{\smalltalcqe}[0]{{\small small}-t{$\cal ALCQ$e}}
\newcommand{\trule}[0]{\xhookrightarrow}
\newcommand{\nodes}[1]{{\cal N}({#1})}
\newcommand{\trans}[1]{{\cal T}({#1})}
\newcommand{\rconts}[1]{\llparenthesis #1 \rrparenthesis} %record contents
\newcommand{\rupd}[2]{{#1}\llparenthesis #2 \rrparenthesis} %record update

\newcommand{\eform}[0]{\mathit{eform}}
\newcommand{\form}[0]{\mathit{form}}
\newcommand{\free}[0]{\mathit{free}}
\newcommand{\exclprop}[0]{\stackrel{\times}{\longrightarrow}}


%%% Local Variables: 
%%% mode: latex
%%% TeX-master: "main"
%%% End: 


% Possilby remove for final version
%\pagestyle{plain}

\begin{document}
\title{The TypeDB Type System}

\author{TBD
% Jon Haël Brenas\inst{1} \and
% Rachid Echahed\inst{2}\orcidID{0000-0002-8535-8057} \and
% Martin Strecker\inst{3}\orcidID{0000-0001-9953-9871}
}
\institute{UPS \and
 CNRS and University of Grenoble, France
}
\maketitle

\begin{abstract}
Trying to understand the TypeDB type system.
\end{abstract}

\keywords{ TBD
% Automated Theorem Proving,
% Modal Logic,
% Graph Transformations,
% Program Verification
}

% REMOVE THE FOLLOWING FOR FINAL VERSION  !!!!
%\def\theHdefinition{\theHtheorem}

%----------------------------------------------------------------------


%----------------------------------------------------------------------
\section{Background}\label{sec:background}

The main sources of information are the following:

\begin{itemize}
\item the TypeDB documentation available on the
  web\footnote{\url{https://docs.vaticle.com/}},
  henceforth referenced as [DOC];
\item a video about knowledge
  graphs\footnote{\url{https://vaticle.com/use-cases/knowledge-graphs}},
  henceforth referenced as [KGV], or for a particular instant (minute, second)
  within this video as [KGV:min:sec] or simply [KGV:min]
\end{itemize}

When the documentation does not
provide a unique answer, we have conducted experiments with TypeDB to
investigate the effective system behavior, see \secref{sec:questions}.


%----------------------------------------------------------------------
\section{The Type System, Conceptually}\label{sec:type_system_conceptually}

The following gives an abstract account of the TypeDB type system.

%......................................................................
\subsection{Structural description}

We distinguish three levels of structural description:
\begin{itemize}
\item Kinds (\secref{sec:structure_kinds}) are predefined concepts of the meta-model
\item Types (\secref{sec:structure_types}) are user-definable entities of a DB
  schema. A type is of a kind.
\item Instances (\secref{sec:structure_instances}) are user-definable entities
  of a DB. An instance is of a type.
\end{itemize}

\subsubsection{Kinds}\label{sec:structure_kinds}

The meta-model is composed of the following elements which we refer to as
\emph{kinds}, also see [KGV:16]:
\begin{itemize}
\item Thing
\item Attribute
\item Entity
\item Relation
\item Role
\end{itemize}
In the following treatment, we ignore rules, which are also mentioned in the
meta-model.

In the meta-model:
\begin{itemize}
\item Attribute, Entity, Relation are related to Thing via a \texttt{sub}
  relation;
\item each of Entity, Attribute, Relation, Role is related to itself via a
  \texttt{sub} relation;
\item Thing is related to Attribute via an \texttt{owns} relation;
\item Relation is related to Role via a \texttt{relates} relation;
\item Thing is related to Role via a \texttt{plays} relation.
\end{itemize}


\paragraph{Kind rules}


$$
\infer{\Gamma \vdash kind(Attribute)}{}
\quad
\infer{\Gamma \vdash kind(Entity)}{}
\quad
\infer{\Gamma \vdash kind(Relation)}{}
\quad
\infer{\Gamma \vdash kind(Role)}{}
$$



\subsubsection{Types}\label{sec:structure_types}

In a schema definition (following the keyword \texttt{define}), new
(attribute; entity; relation; role) types are introduced with the keyword
\texttt{sub}. These subtype relations are gathered in a context. For a subtype
declaration, we write $T' \preceq T$ instead of \texttt{T' sub T}.

\paragraph{Well-formedness rules} The following rules define well-formedness
of a context. The function $names$ retrieves all the newly introduced names of
a context.\remms{TBD}

$$
  \infer{wf(\Gamma, E \preceq T)}{
  wf(\Gamma) & E \notin names(\Gamma)
}
$$


\paragraph{Judgement: Type is of a Kind (rule set~1)}
The following rules derive a judgement $\Gamma \vdash T: K$ saying that type
$T$ is of kind $K$, where $K$ is one of the four kinds of
\secref{sec:structure_kinds}. For example, one could derive: 
$person \preceq entity \vdash person: Entity$.

Base case:
$$
\infer{\Gamma \vdash \mathtt{attribute} : Attribute}{}
\quad
\infer{\Gamma \vdash \mathtt{entity} : Entity}{}
\quad
\infer{\Gamma \vdash \mathtt{relation} : Relation}{}
\quad
\infer{\Gamma \vdash \mathtt{role} : Role}{}
$$

Subtype:
$$
\infer{\Gamma, T' \preceq T \vdash T' : K}{
  wf(\Gamma, T' \preceq T) & \Gamma \vdash T : K 
}
$$

Weakening:
$$
\infer{\Gamma, \Gamma' \vdash T : K}{
  wf(\Gamma, \Gamma')
  & 
  \Gamma \vdash T : K
}
$$


\paragraph{Judgement: Type is of a Kind (rule set~2)}

One may doubt that \texttt{entity} is itself a type (and similarly for
\texttt{attribute} etc.), see the questions in
\secref{sec:questions_type_decls}.

Base case:
$$
\infer{\Gamma, T' \preceq K \vdash T' : K}{
  wf(\Gamma, T' \preceq K)  & kind(K)
}
$$

Subtype:
$$
\infer{\Gamma, T' \preceq T \vdash T' : K}{
  wf(\Gamma, T' \preceq T) & \Gamma \vdash T : K 
}
$$

Weakening:
$$
\infer{\Gamma, \Gamma' \vdash T : K}{
  wf(\Gamma, \Gamma')
  & 
  \Gamma \vdash T : K
}
$$


\subsubsection{Instances}\label{sec:structure_instances}




%......................................................................
\subsection{Semantics}





%----------------------------------------------------------------------
\section{Open and Solved Questions}\label{sec:questions}

%......................................................................
\subsection{Type Declarations}\label{sec:questions_type_decls}


\begin{enumerate}
\item Is it possible to introduce the same name both as an entity and as a
  relation type, something like:

  \begin{alltt}
    Foo sub entity;
    Foo sub relation;
  \end{alltt}

\item Is the type hierarchy acyclic, or is it possible to have
  \begin{alltt}
    C1 sub C2;
    C2 sub C1;
  \end{alltt}
  or to introduce a type several times, like
  \begin{alltt}
    C sub entity;
    C1 sub C;
    C2 sub C;
    D sub C1;
    D sub C2;
  \end{alltt}

\item It is not clear whether \texttt{entity} is indeed a type or only a kind
  in the sense of \secref{sec:structure_kinds}. Syntactically, it does not
  seem possible to declare a relation that has \texttt{entity} as its domain.

  A similar question holds for \texttt{attribute}. Would it be possible to
  write the following?

  \begin{alltt}
  e sub entity,
    owns attribute;
  \end{alltt}

  Probably not, because (semantically) the attribute would not have an
  associated value. A follow-up question to this is if attribute definitions
  can have 0 or $\geq 2$ associated \texttt{value} declarations, such as in
  \begin{alltt}
    a sub attribute;
  \end{alltt}
  (without value),  or
  \begin{alltt}
    a sub attribute,
    value string,
    value long;
  \end{alltt}
  (with several incompatible value declarations).
  
  
\end{enumerate}


%......................................................................
\subsection{Objects and Identities}

\begin{enumerate}
\item We have already tried out to create two entities having identical
  attributes but which become two distinct objects.

  The question is whether there are also multi-relations, i.e. two entities
  can be related by the same relation several times.
\end{enumerate}

%----------------------------------------------------------------------
\bibliographystyle{splncs04}
\bibliography{main}



\end{document}

%%% Local Variables: 
%%% mode: latex 
%%% TeX-master: t
%%% coding: utf-8
%%% End: 
