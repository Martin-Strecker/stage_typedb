\documentclass[runningheads]{llncs}

\input{settings}
\input{defs}

% Possilby remove for final version
%\pagestyle{plain}

\begin{document}
\title{The TypeDB Type System}

\author{TBD
% Jon Haël Brenas\inst{1} \and
% Rachid Echahed\inst{2}\orcidID{0000-0002-8535-8057} \and
% Martin Strecker\inst{3}\orcidID{0000-0001-9953-9871}
}
\institute{UPS \and
 CNRS and University of Grenoble, France
}
\maketitle

\begin{abstract}
Trying to understand the TypeDB type system.
\end{abstract}

\keywords{ TBD
% Automated Theorem Proving,
% Modal Logic,
% Graph Transformations,
% Program Verification
}

% REMOVE THE FOLLOWING FOR FINAL VERSION  !!!!
%\def\theHdefinition{\theHtheorem}

%----------------------------------------------------------------------


%----------------------------------------------------------------------
\section{Background}\label{sec:background}

The main sources of information are the following:

\begin{itemize}
\item the TypeDB documentation available on the
  web\footnote{\url{https://docs.vaticle.com/}},
  henceforth referenced as [DOC];
\item a video about knowledge
  graphs\footnote{\url{https://vaticle.com/use-cases/knowledge-graphs}},
  henceforth referenced as [KGV], or for a particular instant (minute, second)
  within this video as [KGV:min:sec] or simply [KGV:min]
\end{itemize}

When the documentation does not
provide a unique answer, we have conducted experiments with TypeDB to
investigate the effective system behavior, see \secref{sec:questions}.


%----------------------------------------------------------------------
\section{The Type System, Conceptually}\label{sec:type_system_conceptually}

The following gives an abstract account of the TypeDB type system.

%......................................................................
\subsection{Structural description}

We distinguish three levels of structural description:
\begin{itemize}
\item Kinds (\secref{sec:structure_kinds}) are predefined concepts of the meta-model
\item Types (\secref{sec:structure_types}) are user-definable entities of a DB
  schema. A type is of a kind.
\item Instances (\secref{sec:structure_instances}) are user-definable entities
  of a DB. An instance is of a type.
\end{itemize}

\subsubsection{Kinds}\label{sec:structure_kinds}

The meta-model is composed of the following elements which we refer to as
\emph{kinds}, also see [KGV:16]:
\begin{itemize}
\item Thing
\item Attribute
\item Entity
\item Relation
\item Role
\end{itemize}
In the following treatment, we ignore rules, which are also mentioned in the
meta-model.

In the meta-model:
\begin{itemize}
\item Attribute, Entity, Relation are related to Thing via a \texttt{sub}
  relation;
\item each of Entity, Attribute, Relation, Role is related to itself via a
  \texttt{sub} relation;
\item Thing is related to Attribute via an \texttt{owns} relation;
\item Relation is related to Role via a \texttt{relates} relation;
\item Thing is related to Role via a \texttt{plays} relation.
\end{itemize}


\paragraph{Kind rules}


$$
\infer{\Gamma \vdash kind(Attribute)}{}
\quad
\infer{\Gamma \vdash kind(Entity)}{}
\quad
\infer{\Gamma \vdash kind(Relation)}{}
\quad
\infer{\Gamma \vdash kind(Role)}{}
$$



\subsubsection{Types}\label{sec:structure_types}

In a schema definition (following the keyword \texttt{define}), new
(attribute; entity; relation; role) types are introduced with the keyword
\texttt{sub}. These subtype relations are gathered in a context. For a subtype
declaration, we write $T' \preceq T$ instead of \texttt{T' sub T}.

\paragraph{Well-formedness rules} The following rules define well-formedness
of a context. The function $names$ retrieves all the newly introduced names of
a context.


$$
  \infer{wf(\Gamma, E \preceq T)}{
  wf(\Gamma) & E \notin names(\Gamma)
}
$$

$$
\infer{\Gamma \vdash \mathtt{attribute} : Attribute}{}
\quad
\infer{\Gamma \vdash \mathtt{entity} : Entity}{}
\quad
\infer{\Gamma \vdash \mathtt{relation} : Relation}{}
\quad
\infer{\Gamma \vdash \mathtt{role} : Role}{}
$$

$$
\infer{\Gamma, T' \preceq T \vdash T' : K}{
  wf(\Gamma, T' \preceq T) & \Gamma \vdash T : K & kind(K)
}
$$



\subsubsection{Instances}\label{sec:structure_instances}




%......................................................................
\subsection{Semantics}





%----------------------------------------------------------------------
\section{Open and Solved Questions}\label{sec:questions}

%......................................................................
\subsection{Type Declarations}


\begin{enumerate}
\item Is it possible to introduce the same name both as an entity and as a
  relation type, something like:

  \begin{alltt}
    Foo sub entity
    Foo sub relation
  \end{alltt}

\item Is the type hierarchy acyclic, or is it possible to have
  \begin{alltt}
    C1 sub C2
    C2 sub C1
  \end{alltt}
  or to introduce a type several times, like
  \begin{alltt}
    C sub entity
    C1 sub C
    C2 sub C
    D sub C1
    D sub C2
  \end{alltt}
\end{enumerate}


%......................................................................
\subsection{Objects and Identities}

\begin{enumerate}
\item We have already tried out to create two entities having identical
  attributes but which become two distinct objects.

  The question is whether there are also multi-relations, i.e. two entities
  can be related by the same relation several times.
\end{enumerate}

%----------------------------------------------------------------------
\bibliographystyle{splncs04}
\bibliography{main}



\end{document}

%%% Local Variables: 
%%% mode: latex 
%%% TeX-master: t
%%% coding: utf-8
%%% End: 
